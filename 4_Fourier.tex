\section{Fourier-Transformation}
Die Furier-Transformation erlaubt, ein beliebiges Signal in eine Summe von n Sinussignale zu zerlegen.\\
Einige wichtige Eigenheiten sind:
\begin{itemize}
    \item Das Frequenzspektrum einer aperiodischen Funktion ist kontinuierlich im Intervall $]-\infty , +\infty [$
    \item Das Frequenzspektrum einer periodischen Funktion ist diskret.
\end{itemize}
Da es sich bei Bildern um zweidimensionale Funktionen handelt, wird bei diesen die zweidimensionale Fourier-Transformation verwendet.\\
Definition und Rechenregeln: Skript S. 53.
\subsection{2D Diskrete Fourier Transformation (DFT)}
Die zweidimensionale DFT ist das Produkt der beiden eindimensionalen DFTs.
\begin{lstlisting}
%this is the image size
SizeXImage = 150;
SizeYImage = 150;

%we want to choose power of 2 and have image symmetric
SizeFFT = max(2^ceil(log(SizeXImage)/log(2)), 2^ceil(log(SizeYImage)/log(2)));
%calculate DFT (keep size of image)
Image_FFT = fft2(Image, SizeFFT, SizeFFT);
\end{lstlisting}

\subsubsection{Berechnen der Inversen 2D DFT}
\begin{lstlisting}
%caluclate the inverse fft an plot
Inv_FFT = abs(ifft2(Image_FFT_Mask));
\end{lstlisting}

\subsubsection{Mitteln des Spektrums in Matlab}
Zur Darstellung ist es oft besser, die Spektrumswerte von 0Hz in die Mitte zu verschieben. Dafür kann der Befehl \lstinline|fftshift| verwendet werden.

\subsection{Periodische Strukturen}
Auch die 2D DFT kann verwendet werden, um periodische Strukturen zu finden.
Da ein Bild aber nicht periodisch ist (Ränder), enthält die 2D DFT oft unerwünschte Artefakte. Diese können durch eine künstliche ''Periodisierung'' eliminiert werden.\\
Die einfachste Möglichkeit bietet die Anwendung einer Fensterfunktion (engl. Window Function) welche zum Rand des Bild stetig gegen Null abfällt. Ein bekanntes solches Filter ist das \textbf{Hann-Filter}.
\begin{equation}
w_n=\frac{1}{2}*(1-\cos{\frac{2*\pi *n}{N-1}})
\end{equation}
\begin{lstlisting}
%apply Hanning window
Hann = hann(ImgS(1))*hann(ImgS(2))';
Image_FFT_Ham = fft2(double(Image).*Hann);
\end{lstlisting}

\subsection{Unterdrückung periodischer Strukturen}
Periodische Strukturen/Störungen können in einem Bild durch ein \textbf{Notch-Filter} eliminiert werden.
\begin{lstlisting}
%use imtool to find the positions of the frequencies
%%imtool(log(abs(fftshift(Image_FFT))),[0 log(sum(MaxVal))]);
NotchPos = [107 102; 95 143;116 170; 151 160; 165 114; 143 90; 185 140; 70 118];

%apply the filter 
Filter  = NotchFilter( size(Image_FFT), NotchPos, 6*ones(1,size(NotchPos, 1)) );

%shortcut
ImgS = size(Image);

%do the inverse fft
ImageIfft = abs(ifft2(Image_FFT.*fftshift(Filter)));
%chose only the relevant part
ImageFilt = ImageIfft(1:ImgS(1), 1:ImgS(2));
\end{lstlisting}

\subsection{Alisaing}
Aliasing Artefakte werden häufig durch das Dezimieren (engl. Downsampling) von periodischen Strukturen erzeugt. Durch einen Glättungsfilter können diese stark reduziert werden.
\begin{lstlisting}
DownSampling = 5;   % Downsampling-Faktor

%apply anti aliasing filter
Mask = fspecial('average', 2*DownSampling);
%Mask = fspecial('gaussian', 10*DownSampling, 0.5*DownSampling);

%first apply filter
ImageDecimate = imfilter(Image,Mask);
%do sampling
ImageDecimate = ImageDecimate(1:DownSampling:end, 1:DownSampling:end);
\end{lstlisting}

\subsection{Lineare Filter und DFFT}
Eine Faltung zweier Funktionen im Ortsraum entspricht einer Multiplikation der dazugehörigen DFTs im Ortsfrequenzraum.\\
So können mehrere Filter zusammengefasst werden.