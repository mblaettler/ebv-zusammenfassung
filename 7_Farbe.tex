\section{Farbe}
\subsection{Farbräume}
Farben können in verschiedenen Farbräumen dargestellt werden. Dabei unterscheidet man hauptsächlich zwischen Additiven und Substraktiven Farbräumen.
\subsubsection{RGB}
Der RGB (Rot, Grün, Blau) Farbraum ist ein additiver Farbraum in welchem durch Mischen der Farben Rot, Grün und Blau (fast) jede beliebige Farbe erzeugt werden kann.\\
Der RGB-Farbraum wird hauptsächlich zur Bildaufnahme und Wiedergabe an LCD/LED-Monitoren verwendet. Matlab speichert die Farbwerte in einem 3x255-Array.
\begin{lstlisting}
%do a three 3 representation
%extract color planes
Red = Image(:,:,1);
Green = Image(:,:,2);
Blue = Image(:,:,3);
\end{lstlisting}

\subsubsection{CMY}
Für nichtstrahlende Oberflächen können keine additiven Farbräume verwendet werden, da diese selbst keine Wellen aussenden sondern diese nur absorbieren. Dafür werden Cyan, Magenta, Yellow als Filterfarben verwendet.\\
Dieser Farbraum spielt vor allem in der Drucktechnik eine grosse Rolle. Da aber durch die Mischung kein tiefes Schwarz erreicht werden kann, wird als vierte Farbe noch Schwarz hinzugenommen.\\
Die Umwandlung von RGB zu CMYK
\begin{equation}
\begin{bmatrix}
C\\
M\\
Y
\end{bmatrix} =
\begin{bmatrix}
1\\
1\\
1
\end{bmatrix} - 
\begin{bmatrix}
R\\
G\\
B
\end{bmatrix}
\end{equation}

\subsubsection{YCbCr}
Ein weiterer Farbraum ist der YCbCr Farbraum. Dieser hat, nebst vielen anderen Farbräumen die Eigenschaft die Luminanz- (Helligkeits-) von der Chrominanz- (Farb-) Werten zu trennen. (Formel zur Berechnung auf Skript S. 92)
\begin{lstlisting}
%transform to YCbCr space
ImageYCbCr = rgb2ycbcr(Image);

%extract color planes
Y = ImageYCbCr(:,:,1);
Cb = ImageYCbCr(:,:,2);
Cr = ImageYCbCr(:,:,3);
\end{lstlisting}